% First of two papers based on the SPE Comparative Solution Project

%\documentclass{elsarticle}
\documentclass{article}
% package to allow adding affiliation into the title
\usepackage[affil-it]{authblk}
% todonotes allow adding notes into the text as comments
\usepackage{todonotes,amsmath,color}
\usepackage[normalem]{ulem}
\newcommand{\JG}[1]{\todo[color=blue!30, size=\small]{JG: #1}}
\newcommand{\red}{\textcolor{red}}


\begin{document}

\title{Reservoir Simulation using a Deterministic Heterogeneity Permeability Data}


\author{B. Lashore\thanks{Electronic address: \texttt{r01bol14@abdn.ac.uk} (Corresponding author)} , J.L.M.A. Gomes}
\affil{School of Engineering, University of Aberdeen, UK}

\author{P. Salinas, C.C. Pain}\affil{Department of Earth Science and Engineering, Imperial College London, UK}
%\author[UoA]{B. Lashore}\corref{cor1}\ead{r01bol14@abdn.ac.uk}
%\author[IC]{P. Salinas} \author[IC]{C.C. Pain} \author[UoA]{J.L.M.A. Gomes}

%\cortext[cor1]{Corresponding author.}
%\address[UoA]{School of Engineering, University of Aberdeen, UK}
%\address[IC]{Department of Earth Science and Engineering, Imperial College London, UK}

\date \today
\maketitle
%\section*{Abstract}
\begin{abstract}
This paper demonstrates that the Overlapping Control Volume Finite Element Method (OCVFEM) for Reservoir Simulation can make full use of the deterministic heterogeneity data provided without resorting to upscaling. The results from this simulation would be benchmarked against the results from the 10$^{\text{th}}$ SPE Comparative Solution Project (SPE 10 project). Although, only the first case from the SPE 10 project, which involves upscaling and pseudoization methods for an easily computed 2D gas-injection, would be simulated for this paper.

The overlapping control volume finite element method (OCVFEM) formulation, makes use of triangular and tetrahedral finite element pairs to represent the model properties. A P0DG meshing scheme is used for permeability in this simulation therefore the nodal data point for permeability are defined as discontinous data points at the centre of each tethrahedral element. The permeability data for the deterministic heterogeneity is spatially assigned within the simulation domain. And using the deterministic data, 3D interpolation is performed to determine the nodal data for the P0DG meshing scheme.

\red{Yet to obtain any information on the simulation}\\
\red{Try to highlight the preliminary results and what you expect to obtain 'till June/July.}

\sout{Currently available reservoir simulators require a homogeneous data input for a property such as permeability. Hence, upscaling is performed to determine a suitable equivalent value to be used in the simulation.} This novel method of reservoir simulation does not require upscaling but instead makes use of the heterogeneous data available within the simulation domain. This has two advantages. First, it removes the need for a computation for upscaling. Second, flow simulation within the domain is similar to that of a real-life heterogeneous reservoir.

In a future paper, the second problem of the 10th SPE Comparative Solution Project would be simulated with the OCVFEM and higher order meshing schemes would be used. The paper would also make use of stochastic methods for data representation. 
\end{abstract}

\end{document}

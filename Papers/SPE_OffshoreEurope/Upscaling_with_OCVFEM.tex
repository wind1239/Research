% First of two papers based on the SPE Comparative Solution Project

%\documentclass{elsarticle}
\documentclass{article}
% package to allow adding affiliation into the title
\usepackage[affil-it]{authblk}
% todonotes allow adding notes into the text as comments
\usepackage{todonotes,amsmath,color}
\usepackage[normalem]{ulem}
\newcommand{\JG}[1]{\todo[color=blue!30, size=\small]{JG: #1}}
\newcommand{\TL}[1]{\todo[color=green!20, size=\small]{TL: #1}}
\newcommand{\red}{\textcolor{red}}

\begin{document}

\title{Reservoir Simulation using a Deterministic Heterogeneity generated from a discrete Permeability Data}


\author{B. Lashore\thanks{Electronic address: \texttt{r01bol14@abdn.ac.uk} (Corresponding author)} , J.L.M.A. Gomes}
\affil{School of Engineering, University of Aberdeen, UK}

\author{P. Salinas, C.C. Pain}\affil{Department of Earth Science and Engineering, Imperial College London, UK}
%\author[UoA]{B. Lashore}\corref{cor1}\ead{r01bol14@abdn.ac.uk}
%\author[IC]{P. Salinas} \author[IC]{C.C. Pain} \author[UoA]{J.L.M.A. Gomes}

%\cortext[cor1]{Corresponding author.}
%\address[UoA]{School of Engineering, University of Aberdeen, UK}
%\address[IC]{Department of Earth Science and Engineering, Imperial College Londdefineon, UK}

\date \today
\maketitle

\begin{abstract}

 Deterministic heterogeniety for permeability means that the value of permeability is known exactly everywhere within the reservoir. This paper discusses the benefits of using a deterministic heterogeneous data for reservoir simulation, instead of a single equivalent (upscaled)/ homogeneous value. Then it describes a method of creating deterministic heterogeneity. Finally, the solution for Cumulative Oil Produced is benchmarked against that of other participants of the first case from the 10$^{\text{th}}$ SPE Comparative Solution Project (SPE10 project).

The overlapping control volume finite element method (OCVFEM) formulation is capable of using properties with deterministic heterogeneity. It employes a dual consistent pressure-velocity representation in CV and FEM spaces embedded in a novel family of tetrahedral finite element-pairs. The absolute permeability tensor field is represented as discontinuous nodal data points defined at the centre of each tetrahedral element; henceforth denoted as P0DG. The deterministic heterogeneity is spatially assigned within the domain in two stages. First, a suitable mesh size is selected to optimize the use of the discrete data. Then, each nodal data value is determined using 3D interpolation. 

The tetrahedral meshing produced more elements than the given fine hexahedral grid of SPE10 test (100 x 1 x 20), in an effort to optimize the discrete data available. The interpolation scheme used allowed the permeability profile to change gradually between discrete data points. Three different mesh sizes were used to explain the relevance of optimizing the mesh size with respect to the distance between the nearest discrete data points. And, it was shown that a mesh size that is relatively small compared to the distance between the nearest discrete data points produced unnecessarily large computational load for the simulator. The simulation was for an easily computed 2D gas-injection. For he preliminary simulation result, as expected, the flow profile shows a high velocity in the high permeability region and a low velocity in the low permeability region. Therefore, the simulation gives a better understanding of the displacement front and helps to give a realistic view of the movement of the fluids in the reservoir.

This novel method of reservoir simulation does not require upscaling but instead makes use of the heterogeneous data available within the simulation domain. This has two advantages. First, it removes the need for a computation for upscaling. Second, flow simulation within the domain is similar to that of a real-life heterogeneous reservoir.

\end{abstract}

\end{document}

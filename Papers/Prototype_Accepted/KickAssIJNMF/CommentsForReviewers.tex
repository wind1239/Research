
\documentclass[12pts,a4paper,amsmath,amssymb,floatfix]{article}
\usepackage{graphicx,pdfpages}% Include figure files
\usepackage{enumerate}% Align table columns on decimal point
\usepackage{bm,dpfloat}% bold math
\usepackage[pdftex,bookmarks,colorlinks=true,urlcolor=rltblue,citecolor=blue]{hyperref}
\usepackage{amsfonts,amsmath,amssymb,stmaryrd,indentfirst}
\usepackage{times,psfrag,comment}
\usepackage{color}
\usepackage{float}


%%%
%%% Page Format
%%%
\setlength\textwidth      {16.cm}
\setlength\textheight     {22.6cm}
\setlength\oddsidemargin  {-0.3cm}
\setlength\evensidemargin {0.3cm}
\setlength\headheight{14.49998pt} 
\setlength\topmargin{0.0cm}
\setlength\headsep{1.cm}
\setlength\footskip{1.cm}
\setlength\parskip{0pt}
\setlength\parindent{0pt}


%%%
%%% Space between lines
%%%
\renewcommand{\baselinestretch}{1.5}

%%%%%%%%%%%%%%%%%%%%%%%%%%%%%%%%%%%%%%%%%%%
%%%%%%                              %%%%%%%
%%%%%%      NOTATION SECTION        %%%%%%%
%%%%%%                              %%%%%%%
%%%%%%%%%%%%%%%%%%%%%%%%%%%%%%%%%%%%%%%%%%%

% Text abbreviations.
\newcommand{\ie}{{\em{i.e., }}}
\newcommand{\eg}{{\em{e.g., }}}
\newcommand{\wrt}{with respect to}
\newcommand{\lhs}{left hand side}
\newcommand{\rhs}{right hand side}
\newcommand{\frc}{\displaystyle\frac}
\newcommand{\PN}[2][error]{P$_{#1}$DG-P$_{#2}$}
\newcommand{\PNDG}[2][error]{P$_{#1}$DG-P$_{#2}$DG}
\newcommand{\red}{\textcolor{red}}
\newcommand{\blue}{\textcolor{blue}}
%%%%%%%%%%%%%%%%%%%%%%%%%%%%%%%%%%%%%%%%%%%
%%%%%%                              %%%%%%%
%%%%%% END OF THE NOTATION SECTION  %%%%%%%
%%%%%%                              %%%%%%%
%%%%%%%%%%%%%%%%%%%%%%%%%%%%%%%%%%%%%%%%%%%

\begin{document}

\begin{description}
%%%
\item[\blue{Comment 1:}] \blue{``In Section 2.5 I found the interchange of the symbol {\bf v} (force per unit volume) with {\bf u} (saturation-weighted Darcy velocity) to be confusing: in the notation of eqn. 2, I would have expected to find $\mathbf{\tilde{v}_{\alpha}}$ instead of $\mathbf{\tilde{u}_{\alpha}}$. The same goes for eqn. 15. Moreover, it is not clear how eqn. 12 is derived and whether eqn. 15 replaces eqn. 13 or not. The two final paragraphs in the section are rather obscure and should be expanded more.''}\\
 bal bal
%%%
\item[\blue{Comment 2:}] \blue{``The matrix $\mathbf{B}$ and the source term $s_{p}$ in eqn. 10 are not explicitly defined as the terms of eqn. 5 are.''}\\
%%%
\item[\blue{Comment 3:}] \blue{``(...) a statement in the {\it Conclusions} that 'control volumes do not span elements' apparently contradicts the sentence that 'Control volume (CV) boundaries span finite elements where material properties	are defined' (...) ''}\\
%%%
\item[\blue{Comment 4:}] \blue{``(...) the demonstration case `3d fluvial channel' is not sufficiently developed. It is not clear what are the specific challenges that a similar problem poses, compared to similar efforts found in the literature; and it is not specified what is the representative computational time for its solution (...) ''}\\
%%%
\item[\blue{Comment 5:}] \blue{``
''}\\
%%%

%%%
\end{description}

\end{document}

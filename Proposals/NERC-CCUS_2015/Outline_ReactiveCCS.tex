%\documentclass[11pts,a4paper,amsmath,amssymb,floatfix]{article}%{report}%{book}
\documentclass[12pts,a4paper,amsmath,amssymb,floatfix]{article}%{report}%{book}
\usepackage{graphicx,wrapfig}% Include figure files
%\usepackage{dcolumn,enumerate}% Align table columns on decimal point
\usepackage{bm,dpfloat}% bold math
\usepackage[pdftex,bookmarks,colorlinks=true,urlcolor=rltblue,citecolor=blue]{hyperref}
\usepackage{amsfonts,amsmath,amssymb,stmaryrd,indentfirst}
%\usepackage{chemist}
\usepackage{times,psfrag}
\usepackage{natbib}
\usepackage{color}
\usepackage{units}
\usepackage{rotating}
\usepackage{multirow}
\usepackage{gantt}
\usepackage{pdflscape}
 
\usepackage{enumerate}%,enumitem}% Align table columns on decimal point

%\usepackage{pifont}
%\usepackage{subfigure}
%\usepackage{subeqnarray}
%\usepackage{ifthen}
 
\usepackage{supertabular}
\usepackage{moreverb}
\usepackage{listings}
\usepackage{palatino}
%\usepackage{doi}
\usepackage{longtable}
\usepackage{float}
\usepackage{perpage}
\MakeSorted{figure}
%\usepackage{pdflscape}


\definecolor{rltblue}{rgb}{0,0,0.75}


%\usepackage{natbib}
\usepackage{fancyhdr} %%%%
\pagestyle{fancy}%%%%
% with this we ensure that the chapter and section
% headings are in lowercase
%%%%\renewcommand{\chaptermark}[1]{\markboth{#1}{}}
\renewcommand{\sectionmark}[1]{\markright{\thesection\ #1}}
\fancyhf{} %delete the current section for header and footer
\fancyhead[LE,RO]{\bfseries\thepage}
\fancyhead[LO]{\bfseries\rightmark}
\fancyhead[RE]{\bfseries\leftmark}
\renewcommand{\headrulewidth}{0.5pt}
% make space for the rule
\fancypagestyle{plain}{%
\fancyhead{} %get rid of the headers on plain pages
\renewcommand{\headrulewidth}{0pt} % and the line
}

\def\newblock{\hskip .11em plus .33em minus .07em}
\usepackage{color}

%\usepackage{makeidx}
%\makeindex

%\setlength\textwidth      {17.cm}
%\setlength\textheight     {21.6cm}
%\setlength\oddsidemargin  {0.0cm}
%\setlength\evensidemargin {0.0cm}

%\setlength\headheight{14.49998pt} 
%\setlength\topmargin{0.0cm}
%\setlength\headsep{1.cm}
%\setlength\footskip{1.cm}
%\setlength\parskip{0pt}
%\setlength\parindent{0pt}

\usepackage{geometry}
 \geometry{
 a4paper,
 total={210mm,297mm},
 left=20mm,
 right=20mm,
 top=30mm,
 bottom=25mm,
 }



%%%
%%% Headers and Footers
\lhead[] {\text{\small{iCO$_{2}$-EGS}}} 
%\rhead[]{{\text{\small{CoMaELa}}}}
\rhead[]{\text{\small{Reducing Uncertainty in the Integrity of Potential CCS Sites}}}
%\rhead[] {{\text{\small{Reducing Uncertainty in the Integrity of Potential CCS Sites}}}}
%\chead[] {\text{\small{Session 2012/13}}} 
\lfoot[]{{\today}}
%\cfoot[\thepage]{\thepage}
\rfoot[\text{\small{\thepage}}]{\thepage}
\renewcommand{\headrulewidth}{0.8pt}



\usepackage[T1]{fontenc}
\usepackage[utf8]{inputenc}
\usepackage{lmodern}
\usepackage[version=3]{mhchem}

\makeatletter
\newcounter{reaction}
%%% >> for article <<
\renewcommand\thereaction{C\,\arabic{reaction}}
%%% << for article <<
%%% >> for report and book >>
%\renewcommand\thereaction{C\,\thechapter.\arabic{reaction}}
%\@addtoreset{reaction}{chapter}
%%% << for report and book <<
\newcommand\reactiontag{\refstepcounter{reaction}\tag{\thereaction}}
\newcommand\reaction@[2][]{\begin{equation}\ce{#2}%
\ifx\@empty#1\@empty\else\label{#1}\fi%
\reactiontag\end{equation}}
\newcommand\reaction@nonumber[1]{\begin{equation*}\ce{#1}%
\end{equation*}}
\newcommand\reaction{\@ifstar{\reaction@nonumber}{\reaction@}}
\makeatother


%%%
%%% Notes
%%%
\newcommand{\JGnote}[1]{\fbox{\parbox{\textwidth}{\color{blue} JG: #1}}}
\newcommand{\red}{\textcolor{red}}
\newcommand{\blue}{\textcolor{blue}}
\newcommand{\green}{\textcolor{green}}
 

%%%
%%% space between lines
%%%
\renewcommand{\baselinestretch}{1.3}

\newenvironment{VarDescription}[1]%
  {\begin{list}{}{\renewcommand{\makelabel}[1]{\textbf{##1:}\hfil}%
    \settowidth{\labelwidth}{\textbf{#1:}}%
    \setlength{\leftmargin}{\labelwidth}\addtolength{\leftmargin}{\labelsep}}}%
  {\end{list}}

%%%%%%%%%%%%%%%%%%%%%%%%%%%%%%%%%%%%%%%%%%%
%%%%%%                              %%%%%%%
%%%%%%      NOTATION SECTION        %%%%%%%
%%%%%%                              %%%%%%%
%%%%%%%%%%%%%%%%%%%%%%%%%%%%%%%%%%%%%%%%%%%

% Text abbreviations.
\newcommand{\ie}{{\em{i.e., }}}
\newcommand{\eg}{{\em{e.g., }}}
%XS\newcommand{\cf}{{\em{cf., }}}
\newcommand{\wrt}{with respect to}
\newcommand{\lhs}{left hand side}
\newcommand{\rhs}{right hand side}
% Commands definining mathematical notation.

% This is for quantities which are physically vectors.
\renewcommand{\vec}[1]{{\mbox{\boldmath$#1$}}}
% Physical rank 2 tensors
\newcommand{\tensor}[1]{\overline{\overline{#1}}}
% This is for vectors formed of the value of a quantity at each node.
\newcommand{\dvec}[1]{\underline{#1}}
% This is for matrices in the discrete system.
\newcommand{\mat}[1]{\mathrm{#1}}


\DeclareMathOperator{\sgn}{sgn}
\newtheorem{thm}{Theorem}[section]
\newtheorem{lemma}[thm]{Lemma}

%\newcommand\qed{\hfill\mbox{$\Box$}}
\newcommand{\re}{{\mathrm{I}\hspace{-0.2em}\mathrm{R}}}
\newcommand{\inner}[2]{\langle#1,#2\rangle}
\renewcommand\leq{\leqslant}
\renewcommand\geq{\geqslant}
\renewcommand\le{\leqslant}
\renewcommand\ge{\geqslant}
\renewcommand\epsilon{\varepsilon}
\newcommand\eps{\varepsilon}
\renewcommand\phi{\varphi}
\newcommand{\bmF}{\vec{F}}
\newcommand{\bmphi}{\vec{\phi}}
\newcommand{\bmn}{\vec{n}}
\newcommand{\bmns}{{\textrm{\scriptsize{\boldmath $n$}}}}
\newcommand{\bmi}{\vec{i}}
\newcommand{\bmj}{\vec{j}}
\newcommand{\bmk}{\vec{k}}
\newcommand{\bmx}{\vec{x}}
\newcommand{\bmu}{\vec{u}}
\newcommand{\bmv}{\vec{v}}
\newcommand{\bmr}{\vec{r}}
\newcommand{\bma}{\vec{a}}
\newcommand{\bmg}{\vec{g}}
\newcommand{\bmU}{\vec{U}}
\newcommand{\bmI}{\vec{I}}
\newcommand{\bmq}{\vec{q}}
\newcommand{\bmT}{\vec{T}}
\newcommand{\bmM}{\vec{M}}
\newcommand{\bmtau}{\vec{\tau}}
\newcommand{\bmOmega}{\vec{\Omega}}
\newcommand{\pp}{\partial}
\newcommand{\kaptens}{\tensor{\kappa}}
\newcommand{\tautens}{\tensor{\tau}}
\newcommand{\sigtens}{\tensor{\sigma}}
\newcommand{\etens}{\tensor{\dot\epsilon}}
\newcommand{\ktens}{\tensor{k}}
\newcommand{\half}{{\textstyle \frac{1}{2}}}
\newcommand{\tote}{E}
\newcommand{\inte}{e}
\newcommand{\strt}{\dot\epsilon}
\newcommand{\modu}{|\bmu|}
% Derivatives
\renewcommand{\d}{\mathrm{d}}
\newcommand{\D}{\mathrm{D}}
\newcommand{\ddx}[2][x]{\frac{\d#2}{\d#1}}
\newcommand{\ddxx}[2][x]{\frac{\d^2#2}{\d#1^2}}
\newcommand{\ddt}[2][t]{\frac{\d#2}{\d#1}}
\newcommand{\ddtt}[2][t]{\frac{\d^2#2}{\d#1^2}}
\newcommand{\ppx}[2][x]{\frac{\partial#2}{\partial#1}}
\newcommand{\ppxx}[2][x]{\frac{\partial^2#2}{\partial#1^2}}
\newcommand{\ppt}[2][t]{\frac{\partial#2}{\partial#1}}
\newcommand{\pptt}[2][t]{\frac{\partial^2#2}{\partial#1^2}}
\newcommand{\DDx}[2][x]{\frac{\D#2}{\D#1}}
\newcommand{\DDxx}[2][x]{\frac{\D^2#2}{\D#1^2}}
\newcommand{\DDt}[2][t]{\frac{\D#2}{\D#1}}
\newcommand{\DDtt}[2][t]{\frac{\D^2#2}{\D#1^2}}
% Norms
\newcommand{\Ltwo}{\ensuremath{L_2} }
% Basis functions
\newcommand{\Qone}{\ensuremath{Q_1} }
\newcommand{\Qtwo}{\ensuremath{Q_2} }
\newcommand{\Qthree}{\ensuremath{Q_3} }
\newcommand{\QN}{\ensuremath{Q_N} }
\newcommand{\Pzero}{\ensuremath{P_0} }
\newcommand{\Pone}{\ensuremath{P_1} }
\newcommand{\Ptwo}{\ensuremath{P_2} }
\newcommand{\Pthree}{\ensuremath{P_3} }
\newcommand{\PN}{\ensuremath{P_N} }
\newcommand{\Poo}{\ensuremath{P_1P_1} }
\newcommand{\PoDGPt}{\ensuremath{P_{-1}P_2} }

\newcommand{\metric}{\tensor{M}}
\newcommand{\configureflag}[1]{\texttt{#1}}

% Units
\newcommand{\m}[1][]{\unit[#1]{m}}
\newcommand{\km}[1][]{\unit[#1]{km}}
\newcommand{\s}[1][]{\unit[#1]{s}}
\newcommand{\invs}[1][]{\unit[#1]{s}\ensuremath{^{-1}}}
\newcommand{\ms}[1][]{\unit[#1]{m\ensuremath{\,}s\ensuremath{^{-1}}}}
\newcommand{\mss}[1][]{\unit[#1]{m\ensuremath{\,}s\ensuremath{^{-2}}}}
\newcommand{\K}[1][]{\unit[#1]{K}}
\newcommand{\PSU}[1][]{\unit[#1]{PSU}}
\newcommand{\Pa}[1][]{\unit[#1]{Pa}}
\newcommand{\kg}[1][]{\unit[#1]{kg}}
\newcommand{\rads}[1][]{\unit[#1]{rad\ensuremath{\,}s\ensuremath{^{-1}}}}
\newcommand{\kgmm}[1][]{\unit[#1]{kg\ensuremath{\,}m\ensuremath{^{-2}}}}
\newcommand{\kgmmm}[1][]{\unit[#1]{kg\ensuremath{\,}m\ensuremath{^{-3}}}}
\newcommand{\Nmm}[1][]{\unit[#1]{N\ensuremath{\,}m\ensuremath{^{-2}}}}

% Dimensionless numbers
\newcommand{\dimensionless}[1]{\mathrm{#1}}
\renewcommand{\Re}{\dimensionless{Re}}
\newcommand{\Ro}{\dimensionless{Ro}}
\newcommand{\Fr}{\dimensionless{Fr}}
\newcommand{\Bu}{\dimensionless{Bu}}
\newcommand{\Ri}{\dimensionless{Ri}}
\renewcommand{\Pr}{\dimensionless{Pr}}
\newcommand{\Pe}{\dimensionless{Pe}}
\newcommand{\Ek}{\dimensionless{Ek}}
\newcommand{\Gr}{\dimensionless{Gr}}
\newcommand{\Ra}{\dimensionless{Ra}}
\newcommand{\Sh}{\dimensionless{Sh}}
\newcommand{\Sc}{\dimensionless{Sc}}


% Journals
\newcommand{\IJHMT}{{\it International Journal of Heat and Mass Transfer}}
\newcommand{\NED}{{\it Nuclear Engineering and Design}}
\newcommand{\ICHMT}{{\it International Communications in Heat and Mass Transfer}}
\newcommand{\NET}{{\it Nuclear Engineering and Technology}}
\newcommand{\HT}{{\it Heat Transfer}}   
\newcommand{\IJHT}{{\it International Journal for Heat Transfer}}

\newcommand{\frc}{\displaystyle\frac}
\newcommand{\CO}{CO\ensuremath{_{2}}}
\newenvironment{frcseries}{\fontfamily{frc}\selectfont}{}
\newcommand{\textfrc}[1]{{\frcseries#1}}

%\usepackage{enumitem}%
%\newlist{ExList}{enumerate}{1}
%\setlist[ExList,1]{label={\bf Example 1.} {\bf \arabic*}}

%\newlist{ProbList}{enumerate}{1}
%\setlist[ProbList,1]{label={\bf Problem 1.} {\bf \arabic*}}

%%%%%%%%%%%%%%%%%%%%%%%%%%%%%%%%%%%%%%%%%%%
%%%%%%                              %%%%%%%
%%%%%% END OF THE NOTATION SECTION  %%%%%%%
%%%%%%                              %%%%%%%
%%%%%%%%%%%%%%%%%%%%%%%%%%%%%%%%%%%%%%%%%%%

% Cause numbering of subsubsections. 
%\setcounter{secnumdepth}{8}
%\setcounter{tocdepth}{8}

\setcounter{secnumdepth}{4}%
\setcounter{tocdepth}{4}%

\begin{document}


\begin{center}
  {\Large {\bf Progress Towards Sustainable Geothermal Energy:}} \\
  {\Large {\bf Integrated Prediction Tool for Migration and Leakage in CO$_{2}$ Geological Storage using New Generation Modelling}}
%{\Large{\bf Integrated Model Framework for CO$_{2}$-Brine Fractured Enhanced Geothermal System $\left(\text{iCO}_{2}\text{-EGS}\right)$}}
\end{center}
\begin{flushright}
\end{flushright}


\abstract{Environmental and industrial porous media flow is key to a wide range of traditional and emerging applications, including hydrocarbon exploration, geothermal reservoir engineering, nuclear waste repository, groundwater contamination, protection and remediation and \CO\; capture and storage. In this project, research into developing an integrated dynamic prediction tool model will yield a capability to monitor migration of \CO\; plumes in geological storage sites. The generic mathematical model (High-order accurate Control Volume Finite Element Methods / Computational Multi-Fluid Dynamics) will enable detailed investigation of: (a) \CO\; transport in heterogeneous media, (b) \CO-brine-rock interactions (dissolution, precipitation and exsolution due to thermal gradients and brine dissolution in supercritical \CO) and impact on permeability / porosity spatial- and time-variability.  The integrated model will be validated against (a) standard SPE reservoir simulation benchmarks and (b) field data obtained from existing experiments (e.g., Frio Brine Pilot Experiment and XXXX).  We have assembled a team which we believe can successfully carry out the proposed work, combining a wide range of skills encompassing physics, mathematics, earth science and engineering. }


%%%
%%% TRACK RECORD (3 pages) 
%%%
\section{Track Record (3 pages)}

\begin{enumerate}[(a)]
%
\item {\bf University of Aberdeen (UoA), School of Engineering - Environmental and Industrial Fluid Mechanics Group (EnIFluG)\footnote{\href{http://www.abdn.ac.uk/engineering/research/environmental-industrial-fluid-mechanics-122.php}{http://www.abdn.ac.uk/engineering/research/environmental-industrial-fluid-mechanics-122.php}}:} The Group specialises in combined computational methods, laboratory experimentation, and field work on permeable/porous media flows, sediment transport, coastal monitoring, turbulent buoyant jets and large-scale environmental flows. The Group has outstanding experimental facilities, including a range of large water. The latest addition is the petroleum engineering laboratory which represents a \pounds 200k investment in new equipment. The laboratory comprises four independent core-flood rigs with custom-made Hassler cells, an Anton Paar density meter and glass capillary viscometers with thermal bath. In May, UoA was awarded a further \pounds 950k (under the framework of Oil $\&$ Gas Academy of Scotland) for the purchase of equipment related to O$\&$G training. Among the equipment that will be purchased is an X-ray micro-computed tomography (microCT) system for in-situ imaging of fluids in rock samples. \red{make it more general}. \\
{\bf Dr Jefferson Gomes (JG)} leads the development of the computational multi-physics models within Fluidity. He published >25 journal papers on CFD, porous media flows, fluid phase equilibria, nuclear criticality and computational optimisation.\\
{\bf Dr Yukie Tanino (YT)} research combines theory and laboratory experiment to model flow and mass transport in obstructed and/or confined systems, including geological reservoirs, inclined pipes, and artificial wetlands for wastewater treatment. She manages the recently established petroleum engineering laboratory in the School.


%
\item {\bf Imperial College London (IC), Department of Earth Science and Engineering – Applied Modelling and Computation Group (AMCG)\footnote{\href{http://www.imperial.ac.uk/engineering/departments/earth-science/research/research-groups/amcg}{http://www.imperial.ac.uk/engineering/departments/earth-science/research/research-groups/amcg}}:} The Group is is a multi-disciplinary team of 60 scientists and engineers with backgrounds in engineering, mathematics, physics, earth and computational sciences. It specialises in the development of world-class numerical methods for modelling multi-scale multi-physical processes, with application to fracture and fragmentation, porous media and turbulent flows, nuclear criticality safety, industrial and environmental flows, ocean modelling, impact cratering and air pollution. \\
{\bf Prof. Christopher C. Pain (CCP)} is head of AMCG. He has a diverse range of modelling/physics interests including: general CFD (which comprises multiphase flow and geophysical fluid dynamics), nuclear safety, optimisation, numerical research (including parallel solution techniques, error measures and mesh adaptivity) and discretisation techniques10-14. He is the original author of the general-purpose parallel CFD model Fluidity, with mesh adaptivity, multiphase and ocean modelling features and has published 125 journal papers and graduated 33 PhD students. 
%
\end{enumerate}


\clearpage

%%%
%%% CASE FOR SUPPORT (16 pages) 
%%%
\section{Case for Support (max 16 pages)}

%%%
%%% SUBSECTION
%%%
\subsection{Background}

In Carbon Capture, Utilisation, and Storage (CCUS) integrated processes, \CO\; is captured from energy-intensive point sources, mitigated by either chemical conversion or injection into geological formations for either indefinite storage (CCS) or to improve hydrocarbon productivity as part of enhanced oil recovery (\CO-EOR) technologies. Each CCUS process is at very different stages of development and optimisation.

Since the publication of the Intergovernmental Panel on Climate Change Report (IPCC, 2005) the injection of \CO\; in geological formations, became a viable option to mitigate GHG emissions. The European Council's directive 2009/31/EC outlines key conditions that must be met to obtain a geological storage site permit including detailed plans to mitigate \CO\; leakage events and continuous monitoring of the site conditions (e.g., amount and quality of injected \CO, temperature and pressure for assessment of phase behaviour, \CO\; migration paths). Although storage technologies have been heavily utilised in \CO-EOR processes since the 70's, detailed and systematic studies on \CO\; transport and interactions with surrounding rock under reservoir conditions are still incipient. The impact of supercritical \CO\; injection (or any fluid at high- or fracking-pressure conditions) on the geo-environment (including ground-fluid flow within overburden strata and injection/production wells under dynamic stress conditions) has received increasing attention from energy and safety communities.

\CO\; storage requires detailed understanding of the behaviour of fluids in the subsurface.  Plume-scale/long-time fate of the injected \CO\; is determined primarily by large-scale features, most notably spatial heterogeneity in permeability and the presence and distribution of fractures or other high permeability pathways.  Yet the structure of geological formations is highly uncertain even with the most advanced seismic imaging and well logging.  Moreover, exposure to acidic \CO-rich brine is expected to modify the porosity, permeability, and the wettability through dissolution, precipitation and exsolution.  Consequently, the petrophysical properties of a storage site will evolve both during injection and post-operation, possibly over many thousands of years.  

Injected supercritical \CO\; migrates from the injection well to far field regions. With rock and fluid properties changing over the life-cycle of the geological storage, assessing the risk of leakage and quantifying the sensitivity of petrophysical properties of overburden strata (e.g., permeability, porosity, stress etc) to \CO\; time-dependent conditions is crucial for geological site screening and selection. 

A wide array of monitoring technologies originally developed for the oil and gas industry to track fluids movement in the subsurface are now being deployed to monitor underground \CO\; migration and leakage into the cap rock and around the wellbore casing: (a) seismic surveying resulting in reconstructed images of the subsurface allowing \CO\; plumes to be mapped and predicted, (b) grid of smart sensors placed in the borehole (injection, production and monitoring wells) to monitor key parameters (temperature, pressure, salinity, pH), and (c) surface \CO\; sensors.
 
%Flow Simulator: In existing academic- and industry-standard flow simulators, the geological volumetric domain is bounded by prescribed surfaces (top and bottom), and faults are represented by an extra set of surfaces.  The geological volume is then subdivided in geologically-defined zones (bounded by surfaces). Each zone is finally subdivided into grid cells where rock and fluids properties are assigned. In conventional flow simulators, a fixed (and usually coarse) grid resolution is applied to each geological-defined zone due to limited computational resources.  However, in many scenarios, a high-resolution grid is required in specific regions to predict strong flow gradients (e.g., saturation, CO2 concentration), and a lower resolution is acceptable elsewhere.  For example, during water flooding, high-resolution is required close to the migrating waterfront where the saturation gradient is large but, behind the front, the saturation changes gradually and high resolution is not required. A more efficient use of computational effort is to dynamically adapt the computational mesh in space and time (Figure ). This is achieved in our novel reservoir simulator, built on top of the state-of-the-art open-source Fluidity software, under development as part of a collaborative programme between AMCG and the PERM groups at Imperial College (ICL) and the E&IFM group at the University of Aberdeen (UoA). Fluidity is an open-source (LGPL) adaptive finite element model with capabilities in a wide variety of CFD applications1  and has demonstrated impressive parallel scaling2. 

%The proposed project is focused on predicting \CO\; plume migration dynamics under reservoir conditions in complex geological heterogeneity. For the simulation of \CO\; migration it is critically important to accurately represent geological structural characteristics (e.g., facies of the same material property, faults and fractures).  Fully unstructured tetrahedral finite elements have unrivalled geometrical flexibility and are by far the best choice to represent these complex geometries and internal boundaries. Such geological-conforming geometry representation is used with the next generation of models based upon a new family of discontinuous finite element methods. This approach promises a step-change in accuracy from conventional corner-point grids methods in industry-standard flow simulators and represents a novel contribution to CCUS technology. The new model is able to optimally adapt the mesh to represent physics of particular interest, e.g., to focus resolution in the pathways of \CO\; migration.  This model will help quantifying time-scales for supercritical \CO\; migration and potential leakage in geological repositories.

%%%
%%% SUBSECTION
%%%
\subsection{Motivation and Objectives}
\CO\; has been injected in underground geological formations to enhance hydrocarbon productivity (e.g. \CO-EOR). %whilst disposing acidic gasses (mainly H$_{2}$S and \CO) from natural occurance or captured from industrial plants. 
This has helped to prove that geological \CO\; storage is a viable engineering and economic strategy for reducing GHG emissions. However, geological \CO\; storage schemes must comply with UK/EUrequirements (European Council's directive 2009/31/EC) for:
\begin{enumerate}[(i)]
%
   \item Site screening and selection to identify potential geological locations for \CO\; storage;\label{Motivation:a}
%
   \item Site characterisation to assess the engineering and economic feasibility;\label{Motivation:b}
%
   \item Simulation of potential injection strategies to assess (i) pressure and stress distribution and their evolution in time, (ii) plume migration, (iii) oil recovery, (iv) storage capacity and (v) flow rate of injection;\label{Motivation:c}
%
   \item Risk assessment to identify risks of leakage via pathways identified in (\ref{Motivation:c}), to design a monitoring scheme for continuous assessment of these risks, and to develop contingency plans in case of  leakage scenarios;\label{Motivation:d}
%
   \item Site management and operation to ensure that the site is operated in compliance with regulations and to continuously assess the risks associated with the storage site during operation, closure and post-closure stages.\label{Motivation:e} 
%
\end{enumerate}
 
Although the main focus of this proposal is to assess potential \CO\; leakage (\ref{Motivation:d}), the technology developed here can also be applied to assess the storage feasibility of new and existing geological sites (\ref{Motivation:b},~\ref{Motivation:c}).  Risk assessment based on detailed numerical simulations can identify specific risk features and processes (e.g., leaky faults and wellbores, unexpected injection pressure rise, residual saturation trapping) which may help prevent unintended \CO\; migration. Our model framework may also be used during the continuous site monitoring stage (\ref{Motivation:e}) to assess potential risks during injection and storage as the model can use data from field sensors (data assimilation) to help predict the likelihood of \CO\; leakage.

The main objectives of this proposal are: 
\begin{enumerate}[(a)]
%
   \item Investigation of \CO\; plume dynamics (migration and leakage);
%
   \item \red{Investigation of multi-fluid flows through zones of potential rheological weakness in the overburden (e.g., faults and chimneys): solid-fluid interactions and stress-induced fractures formation;}
%
   \item \red{Model uncertainty quantification: sensor network (data assimilation) and ROM};
%
   \item Model quality assurance (monitoring $\&$ verification for risk assessment and environmental impact).
%
\end{enumerate}

{\bf The novelty of this project is in addressing in an integrated way: detailed modelling of \CO\; flows under reservoir conditions (plume migration $\&$ leakage); changes on petrophysical properties (permeability/porosity spatial- and time-variability); solid-fluid interactions (integrity of repositories); and monitoring $\&$ verification technologies.} 

\medskip
We shall consider two integrated length-scales in the CCUS problem: 
\begin{enumerate}[(a)]
  \item \red{flow mechanisms and structures in pore-scale} -- {\bf YT};
  \item flow detachment in field scale \blue{(develop this as a backbone for phase change and flow instability)} -- {\bf JG}.
\end{enumerate}


both the polydispersity and the closure problems by resorting to a self-consistent multiscale modelling approach. We shall consider two (length) scales: a microscopic scale, comparable with the characteristic size of the discrete particles, and a macroscopic scale, comparable with the characteristic length of the gradients that the modeller wishes to capture in the system analysed. We shall employ Lagrangian modelling at the microscale and Eulerian modelling at the macroscale. With the information provided by the Lagrangian simulations, we shall develop constitutive equations that we shall implement in the Eulerian model, experimentally validated for different types of flow by means of advanced imaging techniques (Fig.1). The ultimate goal is to develop a self-standing Eulerian (that is, continuum) model for dense fluid-particle flows, able to correctly describe polydispersity and solvable by means of CFD codes.

\begin{comment}
This proposal comes at a time when the UK nuclear sector is resurgent. It is now widely accepted that one of the ways in which the UK can meet its commitments to reducing CO2 emissions, as well as dealing with its over-reliance on imported fuel supplies, is to replace the current fleet of ageing nuclear reactors. 

Natural and industrial porous media is key to a wide variety of traditional and emerging engineering applications, including but not limited to oil and gas extraction from geological reservoirs, carbon capture and storage, geothermal reservoir engineering, soil sciences, groundwater remediation and protection, biological engineering, food processing, fuel cells, nano-technology, construction engineering, wood processing and printing. 

Global warming due to the accumulating concentration of atmospheric CO2 is being accepted by the scientific and broader community. Amongst a number of CO2 atmospheric emission management considerations is carbon dioxide capture and storage, CCS, related to major coal burning power generation. The operation of CCS requires a close synergy of CO2 capture and CO2 storage. Clearly the most effective capture technology is of no value unless the captured CO2 can be stored. Carbon dioxide is a very soluble gas which makes storing CO2 in large volumes of liquids in porous subsurface reservoirs a serious consideration for captured CO2. Subsurface brine aquifers have a large potential for CO2 storage. Computer based reservoir simulation has been used increasingly by oil companies to predict flow and oil recovery behaviour of reservoirs and help in decision making of project viability. The quality of the simulation process depends on the ability of the simulator to represent the physics and chemistry of the various phenomena occurring in the reservoir and the quality of the various properties characterising the reservoir structure and fluids. The Institute of Petroleum Engineering at Heriot-Watt University will use its oil related reservoir simulation experience and existing commercial simulation software to predict the behaviour of injecting carbon dioxide into brine containing structures in the near vicinity of power generation stations. Predicting the behaviour of CO2 injected into subsurface rocks containing brine requires an understanding of the impact of a wide range of issues; the geology of the formation, the properties of the porous rocks and the fluids, and the change in these properties resulting from interaction with CO2. Reservoir simulation provides an effective platform to evaluate these issues and generate forecasts of CO2 storage potential and distribution. Integrating with and using data from BGS and UoE, the Heriot-Watt team will construct reservoir simulation models, using different levels of characterisation, detail, to simulate specific potential storage sites. The team will examine the impact of a wide range of characterising parameters. Injecting CO2 into brine containing rocks is likely to change rock properties affecting flow behaviour. For example the dissolution of acidic CO2 saturated brine could change the mechanical strength of the rock as well as dissolve minerals, again impacting on flow behaviour. As well as predicting this impact in the reservoir simulation process, the Heriot-Watt team will carry out laboratory based evaluations on rocks, representative of the potential sites, on mechanical strength changes as a result of CO2 brine interaction. Using reservoir condition experimental facilities evaluation of chemical interactions using static soaking and dynamic flow of fluids through rocks (core flooding) and measuring effluent ion concentrations and permeability changes, the HWU team will examine the impact of chemical interaction of carbonated brine with the various rock types provided by BGS. The output from these reservoir simulation based evaluations will be integrated with the other partner outputs and use will be made of 3D visualisation to enable project partners to examine the impact of various issues on the short and long term storage capacity and distribution of the considered subsurface storage sites. The output and 3D simulation will also be used to communicate to other interested stake holders, eg. Scottish Government, DTI, Local Government, EC, other power generators, environmental agencies, and not least the general public, the process and potential of subsurface CO2 storage.

\end{comment}



%%%
%%% SCHEDULING CHART (2 pages) 
%%%
\section{Scheduling Chart (2 pages)}



%%%
%%% JUSTIFICATION OF RESOURCES (4 pages) 
%%%
\section{Justification of Resources (4 pages)}


%%%
%%% PATHWAYS TO IMPACT (2 pages)
%%%
\section{Pathways to Impact (2 pages)}


The high cost of collecting geological and well information means that it is difficult to develop mathematical models to predict key quantities such as maximum rate at which fluids can safely be injected or produced.  Traditional modelling methods are unable to represent complex geological structures and may cost hundreds of millions of pounds on unreliable data.  It is the purpose of this research to develop new tools that will allow for an improved understanding of CO2 geological-based storage. This project will be collaborative primarily with an existing industrially funded project on ”Integrated Simulation for Carbonate Reservoirs“ funded by Shell and Qatar Petroleum through QCCSRC. This programme currently funds two PDRAs dedicated to the development of the new reservoir simulator. In addition the MEMPHIS Program Grant supports development on predictive modelling. Researchers funded by these initiatives represent substantial leveraged funding for this proposal.  It is envisioned that the three projects will have common project meetings, seminars and presentations to industrial sponsors from the energy sector.
The proposed research will fit into the recently-established ICL-based Grantham Institute for Climate Change that funds research in all aspects of climate science, climate change mitigation and policy.   The Grantham Institute will provide a forum for linking the results of this research into policy, allowing the concepts to be disseminated to a much wider audience beyond the specialist scientific community. Additionally, this proposed project also fits into a number of R$\&$D Programmes under the UoA Energy Theme. Drs Tanino and Gomes were recently hired to further enhance research activities in the field of oil and gas engineering at UoA.  The outcome from this project will be integrated into policy making research and further foster collaboration with the energy sector in UK through academics in the UoA’s Schools of Law whose expertise are in the regulatory aspects of CCS. 


%%%
%%% DISSEMINATION 
%%%
\section{Dissemination}

Academic dissemination will be done via publications in international refereed literature and presentations at leading international conferences in the areas of multiphase flows, petroleum engineering, computational fluids dynamics and porous media.
Dissemination will also be extended via the learned groups of which the academics are involved, including the SPE. We will organize annual themed workshop to disseminate the results to the beneficiaries and to establish new collaborations within Norway and the UK. We will also organize yearly induction courses on reservoir simulator for new researchers and users providing tutorial problems for training. Towards the end of the project, we will organize an international workshop dissemination event to showcase the results of this project and present our new modelling framework. The entire team, including the PDRAs, will be involved with organizing a series of outreach events to inform the wider general public. Activities will include public lectures, an interactive website, and general science/engineering activities. 
All software functionalities developed under this project will be licensed under open-source LGP licences, so that they are available for use and external code development by anyone wishing to utilise it for non-commercial purposes.  This will allow academics and research labs around the world to use and edit our code for uses in their own reservoir simulators or any other CFD applications.

 


\end{document}
